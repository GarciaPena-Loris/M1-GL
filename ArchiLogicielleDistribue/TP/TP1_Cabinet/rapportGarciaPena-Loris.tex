\documentclass{article} % Définition de la classe du document (article, report, book, etc.)

% Packages et paramètres supplémentaires
\usepackage[utf8]{inputenc} % Encodage des caractères (UTF-8 recommandé)
\usepackage[T1]{fontenc} % Encodage de la police
\usepackage[french]{babel} % Langue du document (français)
\usepackage{amsmath, amssymb} % Packages mathématiques
\usepackage{graphicx} % Pour inclure des images
\usepackage{cite} % Gestion des citations bibliographiques
\usepackage{hyperref} % Liens hypertextes
\usepackage{listings} % Pour inclure du code source

% Titre du document, auteur et date
\title{Architectures Distribué - Java RMI}
\author{GARCIA PENA Loris}
\date{\today} % Utilisez \date{} pour spécifier une date personnalisée

\begin{document} % Début du contenu du document

\maketitle % Génère le titre du document

% Résumé du document
\begin{abstract}
    Ce rapport présente notre travail réalisé dans le cadre du TP1 portant sur Java RMI (Remote Method Invocation). Le TP1 avait pour principaux objectifs :

    \begin{itemize}
        \item Comprendre la notion d'objet distribué, qui permet à des objets Java de communiquer à travers un réseau.
        \item Utiliser le passage de stub ou d'objet sérialisé, des mécanismes essentiels pour la communication entre objets distants.
        \item Explorer le téléchargement de code via l'utilisation de la codebase, une fonctionnalité importante dans le déploiement d'applications RMI.
    \end{itemize}
\end{abstract}

% Table des matières (générée automatiquement)
\tableofcontents % Table des matières (générée automatiquement)

\newpage % Nouvelle page pour la première section

\section{Une première version simple}

\subsection{Introduction}

La technologie Java Remote Method Invocation (RMI) est un système puissant qui permet à un objet s'exécutant dans une machine virtuelle 
Java d'invoquer des méthodes sur un objet s'exécutant dans une autre machine virtuelle Java distante. Cette capacité de communication à distance 
entre des programmes repose sur l'invocation de méthodes sur des objets distribués appellés stub. 
En d'autres termes, elle offre la possibilité d'interagir avec un objet distant comme s'il était local. 
Cela favorise la construction d'applications réparties en utilisant des appels de méthode au lieu d'appels de procédure, 
simplifiant ainsi le développement d'applications distribuées.

Dans le cadre de ce TP, nous nous plaçons dans le contexte d'un cabinet vétérinaire. 
Chaque patient du cabinet, c'est-à-dire chaque animal, possède une fiche individuelle avec un dossier de suivi médical. 
L'objectif est de créer un système où chaque vétérinaire du cabinet peut accéder aux fiches des patients à distance. 
Nous mettrons en place un serveur et développerons un client pour les vétérinaires. 

Au cours de cette première étape de notre projet, nous allons établir les bases en créant une version simple de notre application, 
en développant une classe "Animal" distribuée, et en configurant un serveur et un client pour la communication. 
Nous débuterons en utilisant un environnement local sans gestion de la sécurité, afin de nous familiariser avec les principes fondamentaux de Java RMI. 
Par la suite, nous enrichirons notre application en ajoutant des fonctionnalités avancées pour répondre aux besoins d'un cabinet vétérinaire moderne.


\newpage % Nouvelle page pour la deuxième section

\section{Classe Cabinet Vétérinaire} % Section 2
% Le contenu de la deuxième section va ici...

\newpage % Nouvelle page pour la troisième section

\section{Création de Patient} % Section 3
% Le contenu de la troisième section va ici...

\newpage % Nouvelle page pour la quatrième section

\section{Téléchargement de Code} % Section 4
% Le contenu de la quatrième section va ici...

\newpage % Nouvelle page pour la cinquième section

\section{Alerte} % Section 5
% Le contenu de la cinquième section va ici...

% Vous pouvez ajouter d'autres sections au besoin...


% Début du contenu principal du document

% Votre contenu va ici...

\end{document} % Fin du document
